\begin{table*}
\scriptsize
\caption{\label{tab:ECVresult}Identified groups of equal items.}

\begin{tabular}{|>{\centering}p{0.27\textwidth}|>{\centering}p{0.15\textwidth}|>{\centering}p{0.2\textwidth}|>{\centering}p{0.2\textwidth}|}
\hline 
Paper identifier \& Description  & Type of CV  & Pairs of equal items  & Groups of equal items\tabularnewline
\hline 
Barney 2009 \cite{Barney2009} Perceived priorities of software product investments
in an ideal situation  & comp.\ HCV & (A2, B4)

(B4, B5)

(B4, C1)

(B5, B15)

(B6, B7)

(B7, B8)

(B14, B15)

(B14, B18)

(B17, B18) & (A2, B4)

(B4, C1)

(B5, B15)

(B6, B7)

(B14, B15)

(B17, B18)\tabularnewline
\cline{2-4}
 & uncomp.\ HCV & (B4, B5)

(B4, B8)

(B5, B15)

(B6, B7)

(B7, B12)

(B14, B15)

(B14, B18)

(B16, B17)

(B12, B13) & (B4, B5)

(B5, B15)

(B6, B7)

(B14, B15)

(B16, B17)

(B12, B13)\tabularnewline
\hline 
Berander 2009 \cite{Berander2009a} Software requirements for course management system  & uncomp.\ \& comp.\ HCV  & (3:2, 3:3) & (3:2, 3:3)\tabularnewline
\hline 
Svahnberg 2008 \cite{Svahnberg2008} The view of academia researchers
on the requirements understandability criteria  & CV & (Development, Verification \& Validation)

(Development, Product Planning 1) & (Development, Product Planning 1)\tabularnewline
\hline
\end{tabular}%
\end{table*}


\subsection{\label{rq3}Identifying Prioritization Items with Equal Priority Using ECV (RQ 3)}
This section presents the results of applying ECV to the industrial and academic CV data as found through the systematic literature review. Six primary studies included the raw prioritization results in the paper itself or referenced online sources where the data was available. To collect the data from the remaining 34 papers, the authors of all papers were contacted.

First, the email addresses provided in the papers were used. If no answer was received authors were searched for using Google, Facebook and LinkedIn. Authors from 11 papers provided us with data to be used in the evaluation of ECV. However, due to confidentiality reasons we can not publish this data directly.

In short, ECV was applied to 27 CV prioritization cases from 14 studies.
In the cases of HCV, ECV was applied two times to the same data to test both compensated and uncompensated priorities. Equal items were detected in three prioritization cases. A summary of the results is presented in Table~\ref{tab:ECVresult} and below follows a summary of each relevant study.

In \cite{Svahnberg2008} a prioritization of requirement understandability criteria is presented.
One of the main findings of the paper is that from an academic viewpoint Development and Verification \& Validation are more important than other criteria.
ECV adds new knowledge to these results.
It shows that Development and Verification \& Validation are equally important, i.e.\ it is not true that either one of the criteria is more important.

A prioritization of software requirements for an academic course management system is presented in \cite{Berander2009a}. ECV detected that two features---Assignment Submission and Assignment Feedback---have the same priority.
If the system is developed in several releases Assignment Submission and Assignment Feedback features can be freely interchanged between the releases and, hence, in this way ECV simplifies release planning.

In \cite{Barney2009} software product investments are prioritized with HCV.
The results of ECV was different for uncompensated and compensated HCV.
When compensated HCV was used ECV detected equal items that belonged to different high level prioritization groups ($A$, $B$ and $C$) indicating that ECV provided a more fine-grained view. In the case of uncompensated HCV, on the other hand, all equal items belonged to one high level prioritization group (group $B$).


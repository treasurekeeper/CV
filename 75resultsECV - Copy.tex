
\begin{table*}
	\scriptsize
\caption{\label{tab:ECVresult}Identified groups of equal items}


\begin{tabular}{
|>{\centering}p{0.27\textwidth}
|>{\centering}p{0.15\textwidth}
|>{\centering}p{0.2\textwidth}
|>{\centering}p{0.2\textwidth}|}
\hline 
Paper identifier \& Description  & Compensation & Pairs of equal items  & Groups of equal items\tabularnewline
\hline
Berander2006\citep{Berander2006} Case 1, prioritization of change management issues & uncomp. & (6:5, 6:6) & (6:5, 6:6)\tabularnewline
\hline 
 & comp. & (6:5, 6:6) & (6:5, 6:6)\tabularnewline
\hline 
Barney2009b\citep{Barney2009b} Perceived priorities of software product qualities today  & uncomp. & (A1, A2)

(A2, A3) & (A1, A2)\tabularnewline
\hline 
 & comp. & (A1, A2)

(A2, A3) & (A1, A2)\tabularnewline
\hline 
 Barney2009b\citep{Barney2009b} Perceived priorities of software product qualities in an ideal situation & uncomp. & (A1, A2)

(A2, A3) & (A1, A2)

\tabularnewline
\hline 
 & comp. & (A1, A2)

(A2, A3) & (A1, A2)\tabularnewline
\hline 
Barney2009a\citep{Barney2009a} Perceived priorities of software product qualities in an ideal situation & uncomp. & (A1, A2)

(A1, A3)

(A2, A3)

(A2, D2)

(A3, D2)

(B2, B3)

(B2, B4) & (A1, A2, A3)

(B2, B3)\tabularnewline
\hline 
 & comp. & (A1, A2)

(A1, A3)

(A2, A3)

(B2, B3)

(B2, B4)

(B3, C4) & (A1, A2, A3)

(B2, B3)

(B3, C4)\tabularnewline
\hline 
Barney2009

\citep{Barney2009} Perceived priorities of software product investments in an ideal situation & uncomp & (A1, A2)

(B6, B7)

(B12, B13)

(B14, B15) & (A1, A2)

(B6, B7)

(B12, B13)

(B14, B15)\tabularnewline
\hline 
 & comp. & (A1, A2)

(B4, B5)

(B6, B7)

(B12, B13)

(B14, B15)

(B4, C1) & (A1, A2)

(B4, B5)

(B6, B7)

(B12, B13)

(B14, B15)

(B4, C1)\tabularnewline
\hline 
Berander2009a Software requirements for course management system & uncomp. & (3:2, 3:3)

(5:5, 5:6)

(4:1, 6:1) & (3:2, 3:3)

(5:5, 5:6)

(4:1, 6:1)\tabularnewline
\hline 
 & comp. & (3:2, 3:3)

(5:5, 5:6) & (3:2, 3:3)

(5:5, 5:6)\tabularnewline
\hline
\end{tabular}%
\end{table*}


%
\begin{table*}
	\scriptsize
\caption{\label{tab:ECVresult-1}Identified groups of equal items}


\begin{tabular}{
|>{\centering}p{0.4\textwidth}
|>{\centering}p{0.3\textwidth}
|>{\centering}p{0.2\textwidth}|}
\hline 
Paper identifier \& Description  & Pairs of equal items  & Groups of equal items\tabularnewline
\hline

Berander2006a \citep{Berander2006a} Case B, Prioritization of decision criteria for the approval of software
change requests  & (Criteria A, Criteria H)  & (Criteria A, Criteria H)\tabularnewline
\hline

Feldt2010 \citep{Feldt2010} Challenges and issues related to verification and validation activities in practice & 
(Q7 - lack of experience in software testing, 

Q8 - lack of understanding among management) &
(Q7, Q8)\tabularnewline
\hline 

Feldt2010 \citep{Feldt2010} Challenges and issues related to efficiency of validation activities &
(Q8 - \textquotedbl{}Development driven\textquotedbl{} instead of \textquotedbl{}Test driven\textquotedbl{},

Q10 - Waiting too long for integration to occur -> Continuous integration!) 

(Q9 - Developers not involved in testing, 

Q11 - Verification experts involved too early in project) &

(Q8, Q10) 

(Q9, Q11)
\tabularnewline
\hline 


Svahnberg2008 \citep{Svahnberg2008} The view of academia students on the requirements understandability
criteria  & (Verification Validation, Product Planning 1)  & (Verification Validation, Product Planning 1)\tabularnewline
\hline 
Svahnberg2008 \citep{Svahnberg2008} The view of academia researchers on the requirements understandability
criteria  & (Development, Verification \& Validation) & (Development, Verification \& Validation)\tabularnewline
\hline 
Hu2006\citep{Hu2006} The relative importance of the criterion in practical situation & (BS, CS)

(BS, Cp)

(CS, Cp)

(Ev, RD) & (BS, CS)

(BS, Cp)

(CS, Cp)

(Ev, RD)\tabularnewline
\hline 


Hu2006 \citep{Hu2006} The relative importance of the criterion in optimal situation & (BS, CS) 

(Ev, RD) & (BS, CS) 

(Ev, RD)\tabularnewline
\hline 

Hu2006 \citep{Hu2006} The second round result of the relative importance of the criterion
in optimal situation

(BS=Business Strategy,

CS=Customer Satisfaction,

Cp=Competitors,

RI=Requirement's issuer,

SF=Software Features,

DC=Development Cost,

CT=Calendar Time,

EC=Extra Cost,

Re=Resource,

AS=After-sale Support,

Cx=Complexity,

Ev=Evolution,

RD=Requirements Dependencies, RV=Requirement Volatility) & (BS, CS)

(Cp, DC)

(Cp, EC)

(Cp, Ev) 

(Cp, RD)

(RI, AS) 

(CT, Cx) 

(EC, Ev) 

(EC, RD)

(Ev, RD) & (Cp, EC, Ev, RD)

(BS, CS)

(RI, AS) 

(CT, Cx) \tabularnewline
\hline 
Regnell2001 \citep{Regnell2001} Software requirements for CASE tool Telelogic & (N2, N5) & (N2, N5)\tabularnewline
\hline
\end{tabular}%
\end{table*}


\subsection{Identifying Prioritization Items with Equal Priority Using ECV (RQ 3).\label{rq3}}

This section presents the results of application of ECV to the industrial
and academic CV result data. The raw CV result data is obtained from
primary studies that are revealed by the systematic review. Six primary
studies included the raw prioritization results in the paper itself
or referenced online sources where data was available. To collect
the data from remaining 33 papers, authors of the papers were contacted.
First, emails provided in the papers were used. If no answer was received authors
were searched for using Google, Facebook, and Linkedin. Unfortunately
this search was not productive. Authors from 15 papers provided us with data to replicate their results and to try out ECV on.
However, due to confidentiality reasons we can not publish his data directly and instead urge interested parties to contact the authors directly.
ECV was applied to 27 prioritization cases from 14 studies.
Equal items were detected in 14 prioritization cases.
A summary of the results of is presented in Table~\ref{tab:ECVresult}.

\citep{Berander2006a} presents a prioritization of decision criteria
for change request approval. The goal of the prioritization is to
improve change management in an organization. ECV detects the equality
of two criteria from a perspective of desire (prioritization case
B, criteria anonymously called A and H, in \citep{Berander2006a}).
This means that both criteria are equally important for rejection
or approval of change requests. Practitioners may choose to use any
of the two criteria, for instance, the easier one.

In \citep{Hu2006} prioritization of requirements selection criteria
is presented. CV is used as research method to determine how the requirements
selection is performed in market driven software development in China.
As a result of prioritization \citep{Hu2006} concludes that in an
ideal situation business strategy and customer satisfaction should
be the most important criteria for requirements selection. ECV enhances
this conclusion by showing that business strategy and customer satisfaction
should be regarded as equal. ECV identifies another group of equal
criteria: competitors, requirements dependencies, and evolution.

In \citep{Svahnberg2008} a prioritization of requirement understandability
criteria is presented. ECV shows that from the viewpoint of academic
students, verification and validation have the same importance as
product planning (making strategic product planning decisions: release
planning, choosing which requirements to dismiss).

Student's distribution function assumes that the data is normally
distributed. The data obtained from the primary studies were tested
for normality using Anderson-Darling statistic. The test resulted
in a conclusion that the data is not normally distributed. Therefore,
a non-parametric version of ECV was created. Student's distribution
function was replaced by the distribution fiction created using kernel
density estimation. ECV with kernel density estimation was applied
to the same data as standard ECV. Regretfully, the non-parametric
variant of ECV failed to detect any items with equal priority. Therefore,
ECV 

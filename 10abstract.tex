
\textbf{Context}.
Prioritization is essential part of requirements engineering, software release planning and many other software engineering disciplines.
Cumulative Voting (CV) is known as relatively simple method for prioritizing requirements on a ratio scale.
Historically, CV has been applied in decision making in government elections, corporate governance, and forestry.
CV prioritization results are special type of data -- compositional data.

\textbf{Objectives}.
The purpose of this study is to aid decision making by collecting knowledge on the empirical use of CV and
developing a method for detecting prioritization items with equal priority.

\textbf{Methods}.
We present a systematic literature review of CV and CV result analysis methods.
The review is based on search in electronic databases and snowball sampling of the primary studies.
Relevant studies are selected based on titles, abstracts, and full text inspection.
Additionally, we propose Equality of Cumulative Votes (ECV) -- a CV result analysis method that identifies prioritization items with equal priority.

\textbf{Results}.
CV has been used in not only in requirements prioritization and release planning but also in software process improvement, change impact analysis, model driven software development, etc.
The review has resulted in a collection of state of the practice studies and CV result analysis methods.
ECV has been applied to 27 prioritization cases from 14 studies and has identified nine groups of equal items in three studies.

\textbf{Conclusions}.
We believe that collected studies and CV result analysis methods can help the adoption of CV prioritization method.
The evaluation of ECV indicates that it is able to detect prioritization items with equal priority.
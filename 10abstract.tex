
\textbf{Context.}
Prioritization is an essential part of requirements engineering, software release planning and many other software engineering disciplines.
Cumulative Voting (CV) is known as a relatively simple method for prioritizing requirements on a ratio scale.
Historically, CV has been applied in decision-making in government elections, corporate governance, and forestry.
However, CV prioritization results are of a special type of data---compositional data.

\textbf{Objectives.}
The purpose of this study is to aid decision-making by collecting knowledge on the empirical use of CV and develop a method for detecting prioritization items with equal priority.

\textbf{Methods.}
We present a systematic literature review of CV and CV analysis methods.
The review is based on searching electronic databases and snowball sampling of the found primary studies.
Relevant studies are selected based on titles, abstracts, and full text inspection.
Additionally, we propose Equality of Cumulative Votes (ECV)---a CV result analysis method that identifies prioritization items with equal priority.

\textbf{Results.}
CV has been used in not only requirements prioritization and release planning but also in e.g.\ software process improvement, change impact analysis and model driven software development.
The review presents a collection of state of the practice studies and CV result analysis methods.
In the end, ECV was applied to 27 prioritization cases from 14 studies and identified nine groups of equal items in three studies.

\textbf{Conclusions.}
We believe that the analysis of the collected studies and the CV result analysis methods can help in the adoption of CV prioritization method.
The evaluation of ECV indicates that it is able to detect prioritization items with equal priority and thus provide the practitioner with a more fine-grained analysis.

\begin{table}
	\center
	\scriptsize
\caption{Extracted data (part 11)}
\scalebox{0.8}{

\begin{tabular}{|>{\raggedright}p{0.03\textwidth} |>{\raggedright}p{0.3\textwidth} |>{\raggedright}p{0.17\textwidth}|}
\hline 
No. & Data item & Extracted data\tabularnewline
\hline
1. & Data extractor & A  \tabularnewline
\hline 
2. & Reference & \citet{Ahl2005} \tabularnewline
\hline 

3. & Title & An experimental comparison of five prioritization methods \tabularnewline \hline 

& \textbf{General information} & \tabularnewline \hline 

4. & Research area & requirements engineering  \tabularnewline \hline 
5. & Study subjects & students \tabularnewline \hline 
6. & Study setting & academic \tabularnewline \hline
7. & Is CV used as research method (research m.) or industry practice (industry p.) & industry p. \tabularnewline \hline 
8. & Type of the study & experiment \tabularnewline
\hline 
9. & Study location & Sweden \tabularnewline
\hline 
 & \textbf{Cumulative voting} & \tabularnewline
\hline 
10. & What is prioritized? & software requirements \tabularnewline \hline 
11. & Number of stakeholders who do CV & 14 \tabularnewline \hline 
12. & Number of prioritization items,

If CV is used, how many items are in each level? & 13 \tabularnewline \hline 
13. & How is CV tailored? & - \tabularnewline \hline 
14. & What methods are used to analyze CV results? & - \tabularnewline \hline 
& \textbf{Quality evaluation} &\tabularnewline \hline 
15. & Study setting rating & 0 \tabularnewline \hline 
16. & Research data availability rating & 0 \tabularnewline \hline 
17. & Rating of correctness of research process & 39 \tabularnewline \hline
\end{tabular}%
}
\end{table}


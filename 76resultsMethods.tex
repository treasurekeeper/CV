\subsection{\label{rq2}CV Result Analysis Methods Identified by RQ 1 (RQ 2)}
The papers identified in the review use various CV result analysis
methods. 
The main goals for CV result analysis are presented in Table \ref{goals_for_methods} and a summary of methods used in the primary studies can be found in Section \ref{analysisMethods}.

In order to present prioritization results many studies use charts or
tables. These charts and tables show the average priority of each prioritization item that is computed from priorities assigned by all stakeholders.
In \citep{Jonsson2005a} a table of five items with highest total priority is presented. 
\citep{Kuzniarz2010} shows tables with $min$, $max$, $\tilde{x}$, $\bar{x}$ and $\sigma$ of priorities assigned by different stakeholders to a particular prioritization item. Finally, in \citep{Rovegard2008,Kuzniarz2010} error bars are added to the chart of final priorities (denoting $\sigma$ of priorities).

In a few cases final priorities are presented in the form of ranks and CV results are degraded from ratio to ordinal scale. This is done when the interest lies only in the order of final priorities.

Several papers are interested in the difference between priorities from
different prioritization perspectives (e.g.\ current and ideal situation)
or stakeholder groups (e.g.\ software developers and management). Pearson
or Spearman correlation coefficients are commonly used to determine what 
the level of similarity is between all priorities from two perspectives.
Whereas, Wilcoxon, Kruskal-Wallis, Nemenyi-Damico-Wolfe-Dunn tests and the 
$\chi^2$ statistic are used to detect if there is a significant
difference in the value of one prioritization item from two or more perspectives. In addition, PCA is used to detect if there are distinct groups of stakeholders with common priorities \citep{Pettersson2008,Chatzipetrou2010,Wohlin2006}.

In some cases, a stakeholder may assign equal priority to several prioritization items or leave several items unrated, e.g.\ the stakeholder may not have carefully considered all prioritization items. Hence, the difference between the items may have been unnoticed.

In \citep{Berander2006a} the scalability of prioritization is measured
using two charts. The first chart shows the average percentages of items given a non-zero value. The second chart shows average percentages of divergence of values. If a stakeholder assigns equal priorities to many prioritization items the divergence of values is low. Unfortunately it is unclear from \citep{Berander2006a} how the average percentage of divergence is calculated.

In \citep{Regnell2000} distribution, disagreement, and satisfaction charts are presented.
The distribution chart shows how the final value of a prioritization
item is constructed from priorities assigned by different stakeholders.
This chart shows how much each stakeholder has contributed to the
final value of a prioritization item.
The disagreement chart shows the level of agreement between different
stakeholders on the value of a particular prioritization item.
The satisfaction chart shows stakeholder satisfaction with prioritization
results by calculating the correlation between final priorities and priorities
assigned by a stakeholder.

The use of bi-plots and ternary plots are proposed in \citep{Chatzipetrou2010}. A bi-plot shows final priorities and stakeholder viewpoints in a two dimensional plane while a ternary plot shows prioritization items inside a triangle. Ternary plots show how many low, medium or high priorities are assigned to a prioritization item. The corners of the triangle represent high, medium, and low priority, e.g.\ if a prioritization item has received mostly high priority values then it is shown closer to the high priority corner.

\begin{table}
	\scriptsize
\caption{Goals for CV result analysis.}
\label{goals_for_methods}

\begin{tabular}{|>{\raggedright}p{0.68\textwidth}|>{\raggedright}p{0.25\textwidth}|}
\hline 
Purpose of the method & Name\tabularnewline
\hline

Show the final priority of each prioritization item. Stakeholder priorities
are combined into one value. & 
Chart or table of final priorities\tabularnewline
\hline 

Difference between priorities assigned by different perspectives (status
quo, ideal situation) or different stakeholder groups (developers,
management) \citep{Chatzipetrou2010}& 
Bi-plot \tabularnewline
\hline 

detect stakeholder groups with similar priorities \citep{Chatzipetrou2010}& Bi-plot \tabularnewline
\hline 

show the relative number of issues that have received high, medium,
or low priority \citep{Chatzipetrou2010}& Ternary plot \tabularnewline
\hline 

detect stakeholder groups with common priorities \citep{Chatzipetrou2010}& PCA \tabularnewline
\hline 

how the final value of prioritization item is constructed from priorities
assigned by different stakeholder. This chart shows how much each
stakeholder has contributed to the final value of prioritization item \citep{Regnell2000}& Distribution chart  \tabularnewline
\hline 

the level of agreement between different stakeholders on value of
particular prioritization item \citep{Regnell2000} & 
Disagreement chart  \tabularnewline
\hline 

satisfaction of a stakeholder with the prioritization results by the
calculating correlation between the final priorities and priorities
assigned by a stakeholder \citep{Regnell2000}& 
Satisfaction chart\tabularnewline
\hline 

percentage of the divergence of the priorities assigned by a stakeholder \citep{Berander2006a} & 
average percentage of divergence\tabularnewline
\hline 
average percentage of items given a non-zero value \citep{Berander2006a} & \tabularnewline
\hline 

detect equal prioritization items (presented in this paper)& 
ECV \tabularnewline
\hline
\end{tabular}
\end{table}

%\subsubsection{\label{codaProblems}Problems with Compositional Data Analysis in Primary Studies}
\subsubsection{\label{codaProblems}Problems with Data Analysis in Primary Studies}

%A few primary studies, as revealed by the systematic review, have problems with the analysis of compositional data.
A few primary studies, as revealed by the systematic review, have problems with the data analysis.

In \citep{Wohlin2006,Pettersson2008} standard PCA is performed without applying log-ratio transformations to compositional data. According to \citep{Aitchison1983}, this is likely to be inadequate and in \citep{Filzmoser2007}, a more appropriate method for performing PCA of compositional data is shown.

The normality of compositional data is defined in \citep{PawlowskyGlahn2007}. It is stated that it is convenient to transform compositional data using isometric log-ratio transformation before the tests for normality can be applied. \citep{Jonsson2005a} violates this requirement by applying the Shapiro-Wilk test for normality to untransformed compositional data.

The Kruskal-Wallis test is used in \citep{Jonsson2005a} to analyze compositional data. The test is used to evaluate the difference between three organization levels. The Kruskal-Wallis test assumes that variables within each sample are independent \citep{Kruskal1952a}. However, values within compositional data vectors are not independent (as described in Section \ref{coda}). Hence, we claim the Kruskal-Wallis test to be somewhat misused in \citep{Jonsson2005a}.
\section{\label{intro}Introduction}

% need for prioritization
Software products are becoming larger and more complex. Each product
is usually affected by a large number of factors such as functional
requirements, quality attributes, or software process improvement
issues. Since time, funding, and resources are limited, it is seldom
possible or even desirable to fully address all the factors. Therefore,
the level of attention to a particular factor should be decided according
to its importance (e.g.\ business value), cost, risk, volatility, 
dependencies between the factors and other such criteria. 
These type of decisions are made by product stakeholders:
users, clients, managers, sponsors, developers, and other persons
associated with the product. In order to make decisions regarding a
large number of factors it is highly advisable to prioritize the factors
in a systematic way \cite{Berander2005}.

% new
Prioritization is commonly used in requirements selection and release planning.
First, project stakeholders prioritize software requirements.
Priority values then can be used to determine the order in which the requirements are going to be implemented.
Requirements with higher priority could be implemented early
while requirements with lower priority may be postponed for later releases or left out.

Another example could be prioritization of potential security threats.
It is done by security professionals, software developers and system administrators
to assess the level of risk and to select risk mitigation activities.
% new

% pros of CV
One of the prioritization methods used in software engineering is Cumulative Voting (CV) \cite{Leffingwell1999}.
The main advantage of CV is that it is relatively simple and fast, yet produces priorities in ratio scale \cite{Berander2005,Ahl2005}.
This allows us to not only determine what prioritization items are more important but also how much more important they are.
(Ratio scale prioritization is particularly important in software release planning and cost-value analysis \cite{Berander2006a, Karlsson1997}.)

% importance of the analysis
Prioritization is usually performed by multiple stakeholders where individual priorities are combined into a single priority list.
Each stakeholder's preferences may have different weight in the final priority.
Such prioritization provides more information than just the priorities of factors.
In the end, it may be useful to analyze the results of the prioritization to assess disagreement between stakeholders, measure stakeholder satisfaction with the results or find distinct groups of stakeholders.

% purpose, problem
The purpose of this study is to help industry practitioners and academia researchers in adopting, using and developing CV, while the importance of prioritization in software engineering and the prospectiveness of CV constitutes a need to do further research in this area.

% methods
This study presents a systematic literature review on the empirical use of CV and CV result analysis methods.
% new
CV results correspond to special type of data---compositional data. Principles of compositional data analysis are described in this paper.
% end new
A new method for CV result analysis, called Equality of Cumulative Votes (ECV), is proposed.
The method identifies prioritization items with \emph{equal} priority.
ECV is evaluated using a considerable amount of data, which was obtained from the primary studies identified by the systematic review (through the kindness of the authors of said studies).

The remainder of this paper is structured as follows. We introduce definitions and place this study in a context in the next section. In Section~\ref{relatedwork} we give a short presentation of related studies. In Section~\ref{methodology} research questions and the methods used in this study are presented. In Section~\ref{slr} the execution of the systematic literature review (SLR) is presented; however, we wait with presenting the results of the SLR. In Section~\ref{ecv} the design of our method of analysis, Equality of Cumulative Votes (ECV), is given, while the results of the SLR and the corresponding evaluation of ECV are presented in Section~\ref{results}. Section~\ref{discussion} provides discussions, presents threats to validity and concludes the paper.

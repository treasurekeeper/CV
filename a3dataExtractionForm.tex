\section{\label{app:dataExtractionForm}Data Extraction Form}

\begin{table}[h!]
\caption{Data extraction form}
\scalebox{0.92}{

\begin{tabular}{|>{\raggedright}p{0.03\textwidth}|>{\raggedright}p{0.3\textwidth}|>{\centering}p{0.1\textwidth}|>{\centering}p{0.5\textwidth}|}
\hline 
No. & Data items & Data type & Description\tabularnewline
\hline
\hline 
1. & Data extractor & Unique identifier & Identification of person who extracted the data\tabularnewline
\hline 
2. & Study identifier & Unique identifier & A paper may include more than one use of CV. Each prioritization is
recorded in separate data record (row in the spreadsheet).\tabularnewline
\hline 
3. & Paper identifier & Unique identifier & Each paper has unique identifier given by reference management software.
Identifier consists of the name of author of the paper and year of
publication.\tabularnewline
\hline 
 & \textbf{General information} &  & \tabularnewline
\hline 
3. & Research area & Narrative & e.g. software engineering, forestry, government elections, corporate
governance\tabularnewline
\hline 
4. & Study subjects & Nominal scale & Possible values: industry professionals, researchers, academia teachers,
academia students, other.\tabularnewline
\hline 
5. & Study scale & Nominal scale & Possible values: industrial, small\tabularnewline
\hline 
6. & Study setting & Nominal scale & Possible values: industrial, academia, unknown\tabularnewline
\hline 
7. & Is CV used as research method or industry practice  & Nominal scale & Possible values: research method, industry practice. Some studies use
CV as a research method in questionnaires while others study it as
industry practice.\tabularnewline
\hline 
8. & Type of the study & Narrative & e.g. survey, case study, experiment\tabularnewline
\hline 
9. & Study location & Narrative & e.g. Spain, Greece\tabularnewline
\hline 
 & \textbf{Cumulative voting} &  & \tabularnewline
\hline 
10. & What is prioritized? & Narrative & software requirements, process improvement issues, software metrics,
etc.\tabularnewline
\hline 
11. & Number of stakeholders who do CV & Absolute scale & \tabularnewline
\hline 
12. & Number of prioritization items,

If CV is used, how many items are in each level? & Absolute scale & \tabularnewline
\hline 
13. & How is CV tailored? & Narrative & \tabularnewline
\hline 
14. & What methods are used to analyze CV results? What is the purpose of
using each analysis method? & Narrative & \tabularnewline
\hline 
 & \textbf{Quality assessment} &  & \tabularnewline
\hline 
15. & Study setting rating & Ordinal scale & See Table~\ref{tab:Study-Setting-Rating}\tabularnewline
\hline 
16. & Research data availability rating & Ordinal scale & See Table~\ref{tab:Research-Data-Availability}\tabularnewline
\hline 
17. & Rating of correctness of research process & Ordinal scale & See Table~\ref{tab:Study-Research-Methodology} and quality evaluation checklist in Appendix~\ref{app:QE}
\tabularnewline
\hline

\end{tabular}
}
\end{table}
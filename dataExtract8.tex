
\begin{table}
	\center
	\scriptsize
\caption{Extracted data (part 8)}
\scalebox{0.8}{

\begin{tabular}{|>{\raggedright}p{0.03\textwidth}|>{\raggedright}p{0.3\textwidth}|>{\raggedright}p{0.17\textwidth}|>{\raggedright}p{0.17\textwidth}|>{\raggedright}p{0.17\textwidth}|>{\raggedright}p{0.17\textwidth}|}
\hline 
No. & Data item & \multicolumn{4}{l|}{Extracted data}\tabularnewline
\hline
1. & Data extractor & A & A & A & A\tabularnewline
\hline 
2. & Reference & \cite{Kuklinski1973} & \cite{Sawyer1962} & \cite{Berander2003} & \cite{Berander2004a} \tabularnewline
\hline 
3. & Title & \raggedright{}Cumulative and Plurality Voting: An Analysis of Illinois'
Unique Electoral System & Game theory and cumulative voting in Illinois: 1902-1954 & Identification of Key Factors in Software Process Management-A Case
Study & Differences in views between development roles in software process
improvement-a quantitative comparison \tabularnewline
\hline 
 & \textbf{General information} &  &  &  & \tabularnewline
\hline 
4. & Research area & government elections & government elections & Software process management & software process improvement\tabularnewline
\hline 
5. & Study subjects & electorate & electorate & professionals & professionals\tabularnewline
\hline 
6. & Study setting & real world & real world & industry & industry\tabularnewline
\hline 
7. & Is CV used as research method (research m.) or industry practice (industry
p.) & industry p. & industry p. & industry p. & industry p.\tabularnewline
\hline 
8. & Type of the study & case study & case study & case study & case study\tabularnewline
\hline 
9. & Study location & USA, Illinois & USA, Illinois & Sweden & Sweden\tabularnewline
\hline 
 & \textbf{Cumulative voting} &  &  &  & \tabularnewline
\hline 
10. & What is prioritized? & election candidates & election candidates & software process improvement issues & software process improvement issues\tabularnewline
\hline 
11. & Number of stakeholders who do CV &  &  & 63 & 63\tabularnewline
\hline 
12. & Number of prioritization items,

If CV is used, how many items are in each level? & 2 &  & 7; five prioritizations of 5, 5, 7, 6, and 5 items & 7; five prioritizations of 5, 5, 7, 6, and 5 items \tabularnewline
\hline 
13. & How is CV tailored? &  & each voter can spend 3 votes on 1, 2 or 3 candidates. &  & \tabularnewline
\hline 
14. & What methods are used to analyze CV results? &  &  &  &

Pearson correlation coefficient between priorities assigned by each two stakeholders
\tabularnewline
\hline 
 & \textbf{Quality evaluation} &  &  &  & \tabularnewline
\hline 
15. & Study setting rating & 3 & 3 & 3 & 3\tabularnewline
\hline 
16. & Research data availability rating & 0 & 0 & 0 & 0\tabularnewline
\hline 
17. & Rating of correctness of research process & 29 & 24 & 30 & 34\tabularnewline
\hline
\end{tabular}%
}
\end{table}



\subsection{\label{coda}Compositional Data Analysis}
CV results can be seen as a special type of data, i.e.\ compositional data.
Compositional data does not contain absolute values. It shows only
the relative weight of a component in a whole. In \citep{Chatzipetrou2010} the authors
propose the use of compositional data analysis for the statistical analysis of CV. 

A compositional data item is a vector ($x$) of positive components with a constant sum $k$:

\begin{equation}
x=(X_{1};\, X_{2};\,\ldots;\, X_{n})\, where\, x_{i}\geq0\, and\,\sum_{j=1}^{n}x_{j}=k.
\label{eq:compositional-data}
\end{equation}

The property of the sum of the items being restricted is called the constant
sum constraint. In CV, priorities assigned by a stakeholder to the items of a prioritization
set is a compositional data vector with a constant sum of 100.
The value of $k$ (i.e.\ 100 in this case) is arbitrary and does not affect the analysis 
of the data because the information is contained in the ratios between the components of 
the vector. The vector can sum up to any number but still hold the same data, i.e.\ vectors 
(1, 2, 7) and (10, 20, 70) are in this case considered equivalent.
This principle is called \emph{scale invariance}.

% new
Another property of compositional data items is \emph{subcompositional coherence}.
Consider that two compositions are analysed. One composition is a subcomposition of the other.
\emph{Subcompositional coherence} means that the results of the analysis are the same for the common parts of the compositions \cite{Aitchison2005}.
This property is important for the analysis of HCV results.
Statements that are made regarding each smaller group of prioritization items are also true for all items prioritized with HCV.
% new 

The priority of an item is relative to the priority of the other items
in the set. Hence, the priority of an individual item is meaningless without
context, i.e.\ the complete set of items. The same item may receive different priority
when put in two different prioritization sets. If the item is put
in a set of items with high priority it will receive a lower relative
priority. This also holds true the other way around i.e.\ if the item is put in a set
with low priority items its priority will be higher.

% section about all the limitations, errors
When doing analysis of compositional data one must take into account that compositional data special type of data
and should be analysed differently than ordinary data.
%Compositional data analysis has, however, serious limitations.
Ordinary unconstrained variables are free to take any positive or negative values,
whereas, compositional data values can only be positive and have a constrained maximum value.
Moreover, components of compositional data vectors are not independent from each other.
The fact that an item is assigned 70 priority points means that the next item can take only values between 0 and 30.
Hence, there is a negative correlation between the items.

Standard parametric statistical tests require that data vectors have multivariate normal distribution.
Vector $X=(X_{1}, X_{2}, \ldots, X_{n})$
is considered to have multivariate normal distribution if any linear
combination of its parts is normally distributed, and linear combination
is defined by:

\begin{equation}
	Y=a_{1}X_{1}+a_{2}X_{2}+\ldots+aX_{n},
\end{equation}

where $Y$ is the product of lineal combination and $a_{i}$ is any
real number. Now, since the sum of priorities assigned in CV must add up
to 100 (or any other constant number) at least one linear combination
of $X$ is not normally distributed because it always adds up to
100:

\begin{equation}
	Y=1\cdot X_{1}+1\cdot X_{2}+\ldots+1\cdot X_{n}=100.
\end{equation}

In our opinion, the above indicates, quite strongly, that CV results do not follow a multivariate normal distribution and, hence, it follows that they should be analysed using non-parametric statistical tests \citep{Pawlowsky-Glahn2006}.

\subsubsection{\label{Problem-of-Zeroes}Problem of Zeroes}
Compositional data analysis requires that log-ratios between any components in a vector can be
computed. But computing a log-ratio with a zero value is, in this case, meaningless. This is
a problem since CV allows stakeholders to assign zero priorities
to some prioritization items (we would even strongly argue that this is very common). 

In compositional data there are two types of zeroes: essential and rounded.
Essential zeroes mean that a data component
is not present. Rounded zeroes mean that the component is present but
its value is very low. We, as others have before us, conjecture that zeroes in CV results are 
rounded because the priority of an item is a completely abstract notion
and the instrument for measuring priority is human judgement \citep{Chatzipetrou2010}.

Before compositional data analysis can be applied to CV results, we should first remove
zeroes in the data. One approach can be to forbid stakeholders
to assign zero priorities. This approach is used in e.g.\ \citep{Pettersson2008}.
But this can add some unnecessary complexity to the prioritization
process and, explicitly, delimits an expert's freedom. 
In \citep{Chatzipetrou2010} the authors propose the use of a multiplicative 
replacement strategy (as defined in \citep{Martin-Fernandez2003}) for CV result analysis.

This method replaces rounded zeroes with small values using the expression
\begin{equation}
r_{j}=\begin{cases}
\delta_{j}, & if\, x_{j}=0,\\
(1-\frac{\sum_{k\mid x_{k}=0}\delta_{k}}{c})x_{j}, & if\, x_{j}>0,\end{cases}\label{eq:zero-replace}
\end{equation}

where $\delta_{j}$ is the imputed value and $c$ is the constant sum constraint.
In order for the total sum of components to stay constant, the equation subtracts 
some value from the items with a priority higher than zero.
More is subtracted from components with higher values than from components with lower 
values (and the value of the imputed $\delta_{j}$ is arbitrary).

\subsubsection{\label{Isometric-Logratio-Transformation}Isometric log-ratio transformation}
In order to apply standard statistical methods to compositional data it should be transformed to remove the inherent correlation of the values.
Compositional data analysis proposes special transformations that change the compositional data values to unconstrained real values.
One such transformation is isometric log-ratio ($ilr$) transformation (as proposed by \citep{Pawlowsky-Glahn2006,Filzmoser2007a}):

\begin{eqnarray}
z & = & \left(z_{1},\,\ldots,\, z_{D-1}\right),\nonumber \\
z_{i} & = & \sqrt{\frac{i}{i+1}}log\frac{\sqrt[i]{\prod_{j=1}^{i}x_{j}}}{x_{i+1}}for\, i=1,\ldots,D-1,\label{eq:ilr}
\end{eqnarray}

where $x$ is the vector that is being transformed and $z$ is the vector that is created. It should be noted that $z$ is shorter than $x$ by one element.

After compositional data vectors are transformed using zero replacement and $ilr$, any standard statistical tests can be applied.

\begin{table}[ht]
\caption{Extracted data (part 9)}
\scalebox{0.8}{

\begin{tabular}{|>{\raggedright}p{0.03\textwidth}|>{\raggedright}p{0.3\textwidth}|>{\raggedright}p{0.17\textwidth}|>{\raggedright}p{0.17\textwidth}|>{\raggedright}p{0.17\textwidth}|>{\raggedright}p{0.17\textwidth}|}
\hline 
No. & Data item & \multicolumn{4}{l|}{Extracted data}\tabularnewline
\hline
1. & Data extractor & A & A & A & A\tabularnewline
\hline 
2. & Reference & \citet{Laukkanen2004}  &  \citet{Barney2008} & \citet{Barney2009a} & \citet{Baca2010}\tabularnewline
\hline 
3. & Title & \begin{raggedright}
Applying voting theory in participatory decision support for sustainable
timber harvesting 
\par\end{raggedright}

& A product management challenge: Creating software product value through
requirements selection & The Relative Importance of Aspects of Intellectual Capital for Software
Companies & Prioritizing Countermeasures through the Countermeasure Method for
Software Security (CM-Sec) \tabularnewline
\hline 
 & \textbf{General information} &  &  &  & \tabularnewline
\hline 
4. & Research area & timber harvesting & release planning, value based requirements engineering & intellectual capital in software company & Software security, online game\tabularnewline
\hline 
5. & Study subjects & professionals & professionals & professionals & researchers\tabularnewline
\hline 
6. & Study setting & industry & industry & industry & industrial\tabularnewline
\hline
7. & Is CV used as research method (research m.) or industry practice (industry
p.) & industry p. & industry p. & industry p. & industry p.\tabularnewline
\hline 
8. & Type of the study & case study & multiple case study & case study & case study\tabularnewline
\hline 
9. & Study location & Finland & Australia, Germany &  & \tabularnewline
\hline 
 & \textbf{Cumulative voting} &  &  &  & \tabularnewline
\hline 
10. & What is prioritized? & timber harvesting alternatives & decision criteria for requirements prioritization & aspects of intellectual capital & potential security attack goals, actors, attack types, and counter-measures\tabularnewline
\hline 
11. & Number of stakeholders who do CV & 7 &  & 32 & unknown\tabularnewline
\hline 
12. & Number of prioritization items,

If CV is used, how many items are in each level? & 7 & 14 & 17 & 
5 items in the highest level, 4 items in second level, 6 items is level 3, 11 items in the lowest level  (total 26 items)
\tabularnewline
\hline 
13. & How is CV tailored? &  &  &  & HCV\tabularnewline
\hline 
14. 
& What methods are used to analyze CV results? 
&  
& 
Tables and charts that display final priorities
&
Correlation matrix between stakeholder groups,

Correlation coefficient between priorities today and ideal situation

Spearmans r correlation coefficient is used 
& \tabularnewline
\hline 
 & \textbf{Quality evaluation} &  &  &  & \tabularnewline
\hline 
15. & Study setting rating & 3 & 3 & 3 & 2\tabularnewline
\hline 
16. & Research data availability rating & 3 & 2 & 2 & 3\tabularnewline
\hline 
17. & Rating of correctness of research process & 22 & 37 & 39 & 24\tabularnewline
\hline
\end{tabular}%
}
\end{table}

\section{Related Work\label{relatedwork}}

%Cumulative voting.
% how can I use CV
Cumulative voting has been studied and applied in various fields.
In forestry it is used to take into account opinions of different parts of society while planning forest harvesting \cite{Laukkanen2004}.
CV is used to vote in government elections \cite{Cooper2010}
and aid decision making in corporate governance \cite{Bhagat1984}.
CV is a part of various software algorithms.
For instance, in \cite{Hoover2003} it is used as part of pattern detection algorithm that is used to locate optic nerve in a retinal image.

In software engineering CV has been applied not only requirements engineering and software release planning \cite{Heikkila2010}
but also in software security \cite{Baca2010}, software quality \cite{Barney2009b}, software metrics \cite{Berander2006}, software process improvement \cite{Pettersson2008}, and software verification and validation \cite{Feldt2010}.

Many studies use CV as a research method.
For instance, in \cite{Svahnberg2008} software impact analysis issues are elicited in structured interviews.
Afterwards the importance of each issue is determined with the help of CV.

% Compositional data 
CV results are compositional data. Principles of compositional data analysis were first defined by J. Aitchison in \cite{Aitchison1982a}.
Two aspects of compositional data particularly important for the present study are
the need to replace zeros in compositional data
and transform the data in order to perform statistical analysis.
Paper \cite{Martin-Fernandez2003} proposes a method to replace zeros and missing values in compositional data.
And an important method for compositional data analysis, $ilr$, is proposed in \cite{Egozcue2003}.

% slr + prioritization methods and cumulative voting
A systematic review of requirements prioritization methods is presented in \cite{Khan2006}. The study focuses on prioritization method comparison and selects eight relevant studies. Two of the studies use CV. These studies are also revealed by the systematic literature review conducted as part of this study. In \cite{Khan2006} the author concludes that there is little research on requirements prioritization and studies usually deal with a small number of requirements.

The systematic literature review presented in this paper does not reveal any CV result analysis methods that allows to identify prioritization items with equal priority. Thus, this problem is not addressed in any way.
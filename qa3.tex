
\begin{table}[ht]
\caption{Quality evaluation results (part 3)}
\scalebox{0.65}{

\raggedright{}\begin{tabular}{
|>{\raggedright}p{0.15\textwidth}
|>{\raggedright}p{0.02\textwidth}|>{\raggedright}p{0.15\textwidth}
|>{\raggedright}p{0.02\textwidth}|>{\raggedright}p{0.17\textwidth}
|>{\raggedright}p{0.02\textwidth}|>{\raggedright}p{0.17\textwidth}
|>{\raggedright}p{0.02\textwidth}|>{\raggedright}p{0.17\textwidth}|}
\hline 
Study identifier & \multicolumn{2}{l|}{\begin{sideways}
Hu2006%
\end{sideways}} & \multicolumn{2}{l|}{\begin{sideways}
Hiltunen2008%
\end{sideways}} & \multicolumn{2}{l|}{\begin{sideways}
Hatton2008%
\end{sideways}} & \multicolumn{2}{l|}{\begin{sideways}
Hatton2007%
\end{sideways}}\tabularnewline
\hline
1. Background, introduction & 3 &  & 3 &  & 3 &  & 3 & \tabularnewline
\hline
2. Problem statement, purpose & 3 &  & 3 &  & 1 &  & 3 & \tabularnewline
\hline
3. Context, independent variables (aka. environment, setting) & 3 &  & 3 &  & 1 & little context provided & 1 & Little context provided\tabularnewline
\hline
4. Related work & 2 &  & 3 &  & 0 &  & 3 & \tabularnewline
\hline
5. Goals and Hypotheses &  &  &  &  &  &  &  & \tabularnewline
\hline
6. Research questions & 3 &  & 0 &  & 0 &  & 0 & \tabularnewline
\hline
7. Design, Research methods & 3 &  & 1 & Authors could have designed a proper experiment to compare the two
voting methods & 1 & It is assumed that requirements in student projects arise with the
same timing as in industry, this data is used to argue which method
for requirements prioritization would be more beneficial in which
phase of student project. The tool that is used to evaluate the requirements
prioritization methods is \textquotedbl{}gut feeling\textquotedbl{}\textquotedbl{}
of the researcher. & 1 & The number of requirements to prioritize is based on the number of
items a human can process rather than typical number of requirements
in real world scenario. Participants are asked to perform several
prioritization methods on the same scenarios, thereby they could have
done the prioritization once and then have reused the priorities from
one method to another.\tabularnewline
\hline
8. Subjects (participants) & 3 &  & 3 &  & 2 &  & 0 & \tabularnewline
\hline
9. Objects & 0 &  & 0 &  & 0 &  & 1 & The study does not present scenarios and goals that are prioritized\tabularnewline
\hline
10. Measures, Data collection procedures, instrumentation & 2 & Questionnaire is not available & 0 & It is unclear how the prioritization was instrumented & 3 &  & 1 & \tabularnewline
\hline
11. Analysis procedure & 3 &  & 2 & Lack of statistical proof & 2 & Lack of statistical proof & 2 & Lack of statistical proof\tabularnewline
\hline
12. Validity threats & 3 &  & 0 &  & 1 &  & 1 & \tabularnewline
\hline
13. Most important findings  & 3 &  & 0 &  & 1 &  & 1 & \tabularnewline
\hline
14. Industry impact, inference, generalisation & 3 &  & 1 &  & 1 &  & 1 & \tabularnewline
\hline
15. Future work & 0 &  & 3 &  & 0 &  & 0 & \tabularnewline
\hline
Total rating & 34 &  & 22 &  & 16 &  & 18 & \tabularnewline
\hline
\end{tabular}%
}
\end{table}


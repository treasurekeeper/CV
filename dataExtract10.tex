
\begin{table}[ht]
\caption{Extracted data (part 10)}
\scalebox{0.8}{

\begin{tabular}{|>{\raggedright}p{0.03\textwidth}|>{\raggedright}p{0.3\textwidth}|>{\raggedright}p{0.17\textwidth}|>{\raggedright}p{0.17\textwidth}|>{\raggedright}p{0.17\textwidth}|>{\raggedright}p{0.17\textwidth}|}
\hline 
No. & Data item & \multicolumn{4}{l|}{Extracted data}\tabularnewline
\hline
1. & Data extractor & A & \multicolumn{2}{l|}{A} & A  \tabularnewline
\hline 
2. & Reference & \citet{Heikkila2010} & \multicolumn{2}{l|}{ \citet{Berander2006}}  & \citet{Barney2009} \tabularnewline
\hline 

3. & Title
& Rigorous Support for Flexible Planning of Product Releases - A Stakeholder-Centric Approach and Its Initial Evaluation 
& \multicolumn{2}{>{\raggedright}p{0.3\textwidth}|}{A goal question metric based approach for efficient measurement framework definition}
& Balancing software product investments
\tabularnewline
\hline 

 & \textbf{General information} &  &  & & \tabularnewline
\hline 
4. & Research area & Software engineering, requirements prioritization, CV as a part of
SCERP framework for software release planning & software change management & software requirements engineering & software engineering \tabularnewline
\hline 
5. & Study subjects & professionals & professionals & professionals & professionals\tabularnewline
\hline 
6. & Study setting & Industrial & industrial & industrial & industrial \tabularnewline
\hline
7. & Is CV used as research method (research m.) or industry practice (industry p.) & industry p. & industry p. & industry p. & industry p.\tabularnewline
\hline 
8. & Type of the study & case study & case study & case study & case study \tabularnewline
\hline 
9. & Study location & - & - & - & -\tabularnewline
\hline 
 & \textbf{Cumulative voting} &  &  &  & \tabularnewline
\hline 
10. & What is prioritized? & features & \multicolumn{2}{l|}{goals and questions} & investments in software development \tabularnewline
\hline 
11. & Number of stakeholders who do CV & 19 & 16 & 19 & 9\tabularnewline
\hline 
12. & Number of prioritization items,

If CV is used, how many items are in each level? & 10 & 7 goals (highest level), 24 questions, 40 metrics (lowest level) & 6 goals (high level items) and 25 questions (low level items) & 5\tabularnewline
\hline 
13. & How is CV tailored? & 
Maximal priority of an item is limited to nine points and the sum of priorities must be 50 priority points &
\multicolumn{2}{l|}{HCV} & \tabularnewline
\hline 
14. & What methods are used to analyze CV results? &
Prioritization result in form of rank of prioritization item &  &
 & \tabularnewline
\hline 
 & \textbf{Quality evaluation} &  &  &  & \tabularnewline
\hline 
15. & Study setting rating & 3 & \multicolumn{2}{l|}{1} & 3  \tabularnewline
\hline 
16. & Research data availability rating & 2 & \multicolumn{2}{l|}{2} & 2  \tabularnewline
\hline 
17. & Rating of correctness of research process & 25 & \multicolumn{2}{l|}{31} & 40\tabularnewline
\hline
\end{tabular}%
}
\end{table}

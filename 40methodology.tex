\section{Methodology\label{methodology}}

This section covers the research questions of this study and the methods used to answer them.

\subsection{Selection of Research Methods}

The main purpose of this study is to collect knowledge on the use of CV in order to help software engineers and researchers in adopting it. This will answer RQ~1 and RQ~2.

One way of collecting this knowledge is to conduct an empirical study. A
survey in a large number of software companies can be used to quantify the
level of adoption of CV in industry (similarly to the study by \citep{Zahedi1986}).
Case studies can be used to receive qualitative feedback on the use of CV
\citep{Runeson2008}.

Knowledge on the empirical use of CV can also be obtained from existing
studies. This may be done by means of a systematic literature review. Several
studies have used CV empirically in industrial as well as in academic settings.
Nevertheless, there are no studies that provide an overview of the current state
of the practice in this field. Therefore, before continuing with the refinement
of CV and conducting new empirical studies (i.e. case study or experiment), a systematic literature review is
required.

This paper proposes a new method for CV result analysis, called Equality of Cumulative Votes (ECV).
(ECV groups prioritization items into groups of items with similar priority.)
As will be presented later, the systematic review did not reveal any methods that solve this problem; however, ECV needs to be evaluated and, hence, applied to CV results.

There are two options to obtain CV results in order to test ECV. One is to conduct a new empirical study. The second option is to collect CV results from existing studies. The latter approach also has the added benefit to try to replicate the results from previous studies and if the CV results from other studies are used, a larger amount of data can be obtained with less effort. Moreover, the generalizability of the evaluation is increasing when prioritization results from different sources and domains are used.
On the other hand, the main benefit of conducting a separate empirical study is the possibility to control the conditions of CV.

In our study we evaluated ECV by obtaining data from previously conducted studies as found by the systematic literature review. In order to obtain the data, authors of relevant primary studies were contacted.

In short, this study consists of two parts: a systematic literature review of CV and an evaluation of ECV.

\subsection{Research Questions}

The systematic review should focus on catching studies that empirically use CV. Information about place, time, scale, and domain of the studies should be collected and the results of the review will hopefully aid academic researchers by identifying paths for further investigation of CV. First research question is:
\begin{description}
\item[RQ 1.] What is the state of practice in empirical studies that use CV?
\end{description}

The level of trust in research results considering CV is determined by the quality of the studies that use CV, hence this study includes an evaluation of the quality of primary studies identified by the systematic review.

Next, a valuable aspect of decision making is the analysis of prioritization results.
Thus, the second research question is:
\begin{description}
\item[RQ 2.] What CV result analysis methods have been presented in papers as identified by RQ 1?
\end{description}

Finally, the evaluation of ECV answers the third research question:
\begin{description}
\item[RQ 3.] Is ECV capable of identifying prioritization items with equal priority?
\end{description}



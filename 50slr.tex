
% Review Method
\section{\label{slr}Systematic Literature Review}

This section presents the design of the systematic literature review. For the results of the execution please see Section~\ref{rq1} and \ref{rq2}.

Table~\ref{tab:reviewActivites} presents an overview of activities performed during the systematic literature review. The review protocol was developed by one researcher and evaluated by another researcher. Studies were searched for in two iterations. The first search was performed by using databases. The second search was performed using snowball sampling \citep{Goodman1961} (snowball sampling examines the references of primary studies revealed by the first search). References that are relevant to the review, i.e.\ they pass the selection criteria, are then added to the set of primary studies.

The search for papers was performed by a single researcher. Study selection, on the other hand, was performed by two researchers. First, one researcher examined all found studies. Next, another researcher re-examined all studies classified as primary studies in addition to 20 randomly selected excluded studies to ensure the quality of the selection.

To ensure the quality of the review, the quality evaluation and data extraction was performed independently by two researchers. Inter-rater analysis was performed using Krippendorf's Alpha statistics \citep{Krippendorff1970,Krippendorff2004a}.

% timetable
\begin{table}
	\scriptsize
\caption{\label{tab:reviewActivites}Review activities.}

\renewcommand{\multirowsetup}{\centering}

\begin{tabular}{
|>{\raggedright}p{0.1\columnwidth}
|>{\raggedright}p{0.5\columnwidth}
|>{\raggedright}p{0.25\columnwidth}
|}
\hline

\multicolumn{2}{|l|}{Review phase} & Researchers involved\tabularnewline
\hline\hline

\multicolumn{2}{|l|}{Trial search in databases} & A \tabularnewline \hline 

\multicolumn{2}{|l|}{Develop review protocol} & A \tabularnewline \hline 

\multicolumn{2}{|l|}{Evaluate review protocol} & B \tabularnewline \hline 

\vspace{0.1cm}\multirow{4}{*}{\begin{sideways}\parbox{2.8cm}{Paper search and selection from databases}\end{sideways}} 

& Search in databases & A \tabularnewline[0.5cm] \cline{2-3}
& Search string validation & A \tabularnewline[0.5cm] \cline{2-3}
& Selection based on metadata & A and B \tabularnewline[0.5cm] \cline{2-3}
& Selection based on full text & A and B \tabularnewline[0.5cm] \hline

\multicolumn{2}{|>{\raggedright}p{0.5\columnwidth}|}{Pilot data extraction (3 papers)} & A \tabularnewline \hline 


\vspace{0.1cm}\multirow{3}{*}{\begin{sideways}\parbox{2.6cm}{Paper selection from the reference lists}\end{sideways}} 

& Selection based on metadata & A and B \tabularnewline[1.2cm] \cline{2-3}
& Selection based on full text & A and B \tabularnewline[1.2cm] \hline 

\multicolumn{2}{|l|}{Data extraction} 	& A and B \tabularnewline  \hline 
\multicolumn{2}{|l|}{Data synthesis} 	& A \tabularnewline  \hline

\end{tabular}\renewcommand{\multirowsetup}{\raggedright}%
\end{table}

% search strategy
\subsection{Data Sources and Search Strategy}
This SLR was designed based on the guidelines by Kitchenham \citep{Kitchenham2007}. First a trial search in electronic databases was conducted. In order to scale the review to a manageable, yet sufficient size, databases were searched with different search strings. Relevant papers that were found during the trial search were used to extract additional search strings. The trial search revealed that the number of studies that use CV is not very large. Therefore, we decided to include not only software engineering studies but also studies in other research areas, such as forestry or corporate governance, since one key aspect we intended to investigate was analysis methods for CV.

Since CV is frequently used in studies without mentioning this in the abstract, full text search in databases is preferable. Unfortunately not all databases support full text search. Full text search was performed in the IEEE Xplore and Springer Link databases. In ACM Digital Library, Inspec\slash Compendex, ISI Web of Knowledge, and SCOPUS only metadata was searched.
Search strings consisting of a Boolean expression (A or B or C or D or E or F or G), where:

\begin{table}
	\scriptsize
\begin{tabular}{
>{\raggedright}p{0.3\columnwidth}
>{\raggedright}p{0.4\columnwidth}
}
A -- Cumulative voting & E -- hundred dollar method \tabularnewline
B -- 100 dollar method & F -- hundred dollar test \tabularnewline
C -- 100 dollar test & G -- hundred point method \tabularnewline
D -- 100 point method & \tabularnewline
\end{tabular}
\end{table}

Search strings contained only synonyms of CV and they did not limit the research area to software engineering. The search was performed independently using each of the search strings in each database. 
All search results were combined and documented using reference management software. The quality of the search strings and the selection of electronic databases were validated against a previously known core set of papers---\citep{Ahl2005,Berander2006,Chatzipetrou2010,Regnell2001}---checking that all papers from the core set were found by the search.

% study selection
\subsection{Study Selection}
To select relevant papers a set of criteria were designed. The criteria for paper selection are presented in Tables~\ref{tab:Paper-search-and} and \ref{tab:Paper-Selection-from}.

Papers were selected in two phases: based on metadata and based on full text.

Obviously, the main criterion for inclusion of a paper is that it must present empirical use of CV or present an analysis of the results of using CV. However, there are papers that pass this criterion but are not relevant for this review. CV is frequently used in computer algorithms. There is a significant difference between the way that humans and computers make decisions. Since this review in concerned with human decisions we excluded papers that present CV that is not performed by humans. In addition, only papers that were written in English were selected and duplicate studies were automatically excluded by the citation management software used in this review.

% Snowball
%Table~\ref{tab:Paper-Selection-from} shows the results of the second paper selection.

% TABLE paper search db
\begin{table}
	\scriptsize
\caption{\label{tab:Paper-search-and}Paper search and selection in the databases.}

\begin{tabular}{|>{\raggedright}p{0.25\columnwidth}|>{\raggedright}p{0.49\columnwidth}|>{\raggedright}p{0.15\columnwidth}|}
\hline
Selection phase & Inclusion criteria & Number of papers selected\tabularnewline
\hline \hline

Search in databases & published from 2001 until 2011 (databases last accessed Feb.\ 20, 2011) & 256 \tabularnewline
\cline{2-2}
& contains search strings & \tabularnewline
\hline 

Selection based on metadata & exclude duplicates and tables of contents& 177 \tabularnewline
\cline{2-2}
&  written in English  & \tabularnewline
\hline

Selection based on full text & full text is available & 127 \tabularnewline
\cline{2-3}
& study involves empirical use of CV or presents analysis of empirical use of CV & 58 \tabularnewline
\cline{2-3}
& CV is done by humans and not software & 25 \tabularnewline
\hline
\end{tabular}%
\end{table}

% TABLE paper search snowball

%
\begin{table}
	\scriptsize
\caption{\label{tab:Paper-Selection-from}Paper selection from the reference lists of the selected papers.}

\begin{tabular}{|>{\raggedright}p{0.25\columnwidth}|>{\raggedright}p{0.49\columnwidth}|>{\raggedright}p{0.15\columnwidth}|}
\hline 
Selection phase & Inclusion criteria & Number of papers selected\tabularnewline
\hline\hline
Selection from references & papers included in the reference lists of relevant papers found in databases & 467 \tabularnewline
\hline 


Selection based on metadata & written in English & 462 \tabularnewline
\cline{2-3}
& reference is already revealed by search in databases & 450 \tabularnewline
\hline
Selection based on full text & full text is available  & 329 \tabularnewline
\cline{2-3}
& study involves empirical use of CV or presents analysis of empirical use of CV & 15 \tabularnewline
\cline{2-2}
& CV is done by humans and not software & \tabularnewline
\hline


\end{tabular}%
\end{table}

% QA

\subsection{\label{QE}Quality Evaluation}
The goal of quality evaluation is to determine the best primary studies according to some measure of quality.
Since the number of studies that use CV is not large, quality evaluation was not used as an exclusion criterion.

Study quality obviously depends on the correctness of the study process including planning, operation, analysis and interpretation of the results (is the study right?) The correctness of the process can be measured by evaluating the description of the study or replicating the study. Thus, to gain the trust of industry practitioners and other researchers, the process of the study must be rigorously described. In short, the description must facilitate replication of the study as well as the presentation of limitations and validity threats.

Even the most correct and rigorously described study is useless if it does not contribute to the industry or research community (is it the right study?) The topic of the research ought to address important goals and issues. The findings of the study should also be significant, i.e.\ there must be a high probability of the results of the study being true. The significance of the findings depends on how realistic the study is, the correctness of the process and the results of the study, as well as the statistical significance of the findings.

% realism
\textbf{Realism} of a study depends on the context, scale, and subjects of the study.
% setting
The study should be conducted in a \textbf{setting} that is similar or equal to the setting in which the findings of the study are intended to be used. Hence, studies that are conducted in an industrial setting are in many cases valuable.
% subjects
The \textbf{subjects} of a study should be similar to the people who are supposed to use the findings of the study. The subjects ought to have appropriate work experience, role in the organization, skills, cultural background, motivation, and so forth.
% scale
The \textbf{scale} of a study refers to the size of the study objects. 
%The object of prioritization is the set of prioritization items.
In the case of this systematic review the scale of a study is measured as the number of prioritization items.
Study in academia may have a large number of prioritization items. At the same time, an industrial study, with professionals as subjects, may involve a smaller number of prioritization items.

Each study may have a different level of realism. Some studies involve industry practitioners in an academic setting to simulate real word practice in a laboratory environment.
Other studies may involve academic researchers that execute a project. For example, researchers may be developing open source software.
On the reality scale these studies are somewhere in between the purely academic and industrial studies.

% the study type
The \textbf{type} of the research study can be considered as a criterion for the evaluation of study realism. \citep{Kitchenham2004} suggest that study designs that are more rigorous (e.g.\ experiments) are more realistic than observational studies (e.g.\ case study) due to a higher level of control. On the other hand \citep{Ivarsson2010} rate study designs based on other criteria, i.e.\ how frequently each type of study design is used in an industrial or academic setting. If a study design is used more in an industrial setting, then it is considered more realistic. For instance, in software engineering, case studies are frequently used in industrial settings, whereas, experiments are usually performed in academia using students as subjects. Therefore, \citep{Ivarsson2010} argue that case studies are more realistic than  formal experiments. Obviously the effect of study design on the study realism may be interpreted in different ways. Therefore, we will not use this parameter in our quality evaluation.

The statistical significance of the results of a study can be used to evaluate the significance of the study findings.
This measure will not be used, because the studies that are evaluated belong to very different research areas, i.e.\ the significance levels of the findings of the studies are not directly comparable for meta-analysis.
Additionally, sometimes no result is more interesting than a significant result.
If study results do not conform to the expectations of researchers, this may reveal important gaps in existing knowledge.
%Nevertheless, the evaluation of the correctness of the study process verifies that the statistical analysis is performed and significance levels are reported.

The ultimate goal of research, at least in software engineering, is in many cases industry impact. However, most of the time ideas need to be developed and validated in academia before industry professionals will risk to adopt them. Therefore, academic impact is important as well. Academic impact is usually measured by the number of citations. Academic impact is also measured for particular researchers, using the number of papers she has published and the number of citations of her papers.
This measure will not be used in our quality evaluation because it is somewhat biased. The number of citations is likely to be lower for newer papers and the number of papers that a researcher has published gives little information about the actual quality or impact of her research.

\subsubsection{Rating of the Studies}
The quality evaluation in our review is based on the evaluation of: ($i$) Study realism. ($ii$) Study scale. ($iii$) Availability of raw results of CV. ($iv$) Quality of the research methodology.

Realism of the studies is rated in three aspects: subjects, setting, and scale.
The subjects and setting is rated according to Table~\ref{tab:Study-Setting-Rating}. The total rating of study realism is determined by summing up the ratings of the two aspects. For instance, if a study is conducted with industry professionals as subjects in an academic context the study will receive rating 1 (out of 2 maximal points). %rto: out of how many points? Out of 100? ;)
%kri: fixed, I guess

In order to rate the scale of a study the number of prioritization items was counted.
If a paper presents several prioritization cases only the prioritization with the largest number of the prioritization items is considered.
If HCV is used all of the prioritization items on different levels are counted together. However, if an item is present in several groups in the hierarchy it is counted only once.

The availability of raw results of CV is rated separately because it is especially important for our purposes (and for most other researchers in order to replicate a study). The data availability rating criteria is given in Table~\ref{tab:Research-Data-Availability}. If the data of a study are not available it is not possible to validate the results of the study and, hence, the credibility of the findings is lower. Ideally the data collected in the study should be presented directly in the paper. An alternative may be to make the data freely available online and reference the online source.

The quality of the research methodology of a paper is rated according to a checklist presented in \ref{app:QE}. The checklist is based on guidelines for presenting research studies as presented in \citep{Wohlin2000,Jedlitschka2005} and the guidelines for quality evaluation of research studies presented in \citep{Ivarsson2010,Kitchenham2007}. Evaluation is done with regard to the rigor of the description and correctness of the research process and reasoning. Checklist items represent issues that research studies should implement and present in research paper. The checklist also contains item descriptions or questions that are used to evaluate the quality. Each item in the checklist is rated according to criteria presented in Table~\ref{tab:Study-Research-Methodology}. The final rating of correctness of the research process of a study is computed by summing up the ratings assigned to all items in the checklist.

Study rating criteria was validated during a trial data extraction. Two researchers each rated three randomly selected papers. Afterwards, differences in ratings were discussed and study rating criteria were updated to avoid differences in interpretation.

% reality level
\begin{table}
	\scriptsize
\caption{\label{tab:Study-Setting-Rating}Rating of study reality level}
\begin{tabular}{|>{\centering}p{0.2\columnwidth}|>{\centering}p{0.3\columnwidth}|>{\centering}p{0.39\columnwidth}|}
\hline 
Aspect & Contribute to relevance (rating 1) & Do not contribute to relevance (rating 0)\tabularnewline
\hline\hline
Subjects & Industry professionals & Academia students or teachers, or other\tabularnewline
\hline 
Context & Industrial & Academia\tabularnewline
\hline 
%Scale of the Study & Industrial scale & Down-scaled industrial, toy examples\tabularnewline
%\hline
\end{tabular}%
\end{table}

% data availability
\begin{table}
	\scriptsize
\caption{\label{tab:Research-Data-Availability}Research data availability rating}
\begin{tabular}{|>{\centering}p{0.1\columnwidth}|>{\raggedright}p{0.83\columnwidth}|}
\hline 
Rating & \centering{}Study rating criteria\tabularnewline
\hline \hline
0 & CV results was not provided in the paper and we was unable to obtain
the results from the authors.\tabularnewline
\hline 
1 & CV results are not provided in the paper but the data was obtained
from the authors. Part of the data is lost or corrupted.\tabularnewline
\hline 
2 & CV results are not provided in the paper but all the data was obtained
from the authors.\tabularnewline
\hline 
3 & All CV results are included in the paper or reference is given to
online source where all the data can be accessed.\tabularnewline
\hline
\end{tabular}%
\end{table}

% correctness
\begin{table}
	\scriptsize
\caption{\label{tab:Study-Research-Methodology}Rating of correctness of research process}
\begin{tabular}{|>{\centering}p{0.1\columnwidth}|>{\raggedright}p{0.83\columnwidth}|}
\hline 
Rating & \centering{}Study rating criteria\tabularnewline
\hline \hline
0 & No description provided.\tabularnewline
\hline 
1 & Only basic information is provided about the checklist item. Or significant
validity threats exist with regard to this item.\tabularnewline
\hline 
2 & Description is sufficient. Some minor questions are left unanswered.
Validity threats may exist but they are not likely to affect the results
of the study.\tabularnewline
\hline 
3 & Description is rigorous and clear. Questions presented in quality evaluation checklist in \ref{app:QE} are answered. Decisions of the study are well
justified, alternatives are discussed. No unhandled validity threats
can be identified.\tabularnewline
\hline
\end{tabular}%
\end{table}

As a result of the rating each study was assigned four rating values on an ordinal scale. In order for us to perform a more advanced analysis of the quality evaluation results these ratings were then converted into ratio scale ranks. For each study, the number of studies that have received lower ratings is counted. The resulting number is the rank of the study; thereby, the quality of a study is expressed as four rank values.

An example of rating values is shown in Table~\ref{tab:Example-of-rating}. Table~\ref{tab:Example-of-ranking} shows ranking values computed for the studies in Table~\ref{tab:Example-of-rating}. 
We can observe that study realism level rating for ST3 is 0. There are no studies that have a lower study realism. Therefore, realism ranking for ST3 is 0. ST1 on the other hand has the highest realism rating. Since ST1 has higher reality level than both ST2 and ST3 it is assigned reality level rank 2.

% example of rating
\begin{table}
	\scriptsize
\caption{\label{tab:Example-of-rating}Example of rating values}

\begin{tabular}{|>{\centering}p{0.1\columnwidth}|>{\centering}p{0.18\columnwidth}|>{\centering}p{0.18\columnwidth}|>{\centering}p{0.18\columnwidth}|>{\centering}p{0.18\columnwidth}|}
\hline 
Study & Realism & Research data availability & Correctness of research process & Number of prioritization items \tabularnewline
\hline \hline
ST1 & 2 & 0 & 15 & 6\tabularnewline
\hline 
ST2 & 1 & 3 & 20 & 69\tabularnewline
\hline 
ST3 & 0 & 3 & 10 & 6\tabularnewline
\hline
\end{tabular}%
\end{table}

% example of ranks
\begin{table}
	\scriptsize
\caption{\label{tab:Example-of-ranking}Example of ranking values}

\begin{tabular}{|>{\centering}p{0.1\columnwidth}|>{\centering}p{0.18\columnwidth}|>{\centering}p{0.18\columnwidth}|>{\centering}p{0.18\columnwidth}|>{\centering}p{0.18\columnwidth}|}
\hline 
Study & Reality level & Research data availability & Correctness of research process & Number of prioritization items \tabularnewline
\hline \hline

ST1 & 2 & 0 & 1 & 0\tabularnewline
\hline 
ST2 & 1 & 1 & 2 & 2\tabularnewline
\hline 
ST3 & 0 & 1 & 0 & 0\tabularnewline
\hline
\end{tabular}%
\end{table}

% data extraction
\subsection{\label{Data-extraction}Data Extraction}
The goal of data extraction is to understand how and why CV is used and how CV results are analyzed in research studies. Ultimately, this will allow us to answer the first and second research questions in our study.

Data extraction was documented with the help of spreadsheet software. Extracted data items are available from \cite{Rinkevics2011a}.

\begin{table}
	\center
	\scriptsize
\caption{Extracted data (part 5)}
\scalebox{0.8}{

\begin{tabular}{|>{\raggedright}p{0.03\textwidth}|>{\raggedright}p{0.3\textwidth}|>{\raggedright}p{0.17\textwidth}|>{\raggedright}p{0.17\textwidth}|>{\raggedright}p{0.17\textwidth}|>{\raggedright}p{0.17\textwidth}|}
\hline 
No. & Data item & \multicolumn{4}{l|}{Extracted data}\tabularnewline
\hline
1. & Data extractor & A & A & A & A\tabularnewline
\hline 
2. & Reference & \cite{Wohlin2006} & \cite{Cooper2010} & \cite{Cole1990}  & \cite{Berander2009a} \tabularnewline
\hline 
3. & Title & Criteria for selecting software requirements to create product value:
An industrial empirical study & A comparison of cumulative voting and generalized plurality voting & Cumulative Voting in a Municipal Election: A Note on Voter Reactions
and Electoral Consequences & Evaluating two ways of calculating priorities in requirements hierarchies
- An experiment on hierarchical cumulative voting \tabularnewline
\hline 
 & \textbf{General information} &  &  &  & \tabularnewline
\hline 
4. & Research area & requirements engineering & government elections & government elections & requirements prioritization\tabularnewline
\hline 
5. & Study subjects & professionals & electorate & electorate & students\tabularnewline
\hline 
6. & Study setting & academia & real word & academia & academia\tabularnewline
\hline 
7. & Is CV used as research method (research m.) or industry practice (industry
p.) & research m. & industry p. & industry p. & research m.\tabularnewline
\hline 
8. & Type of the study & case study & case study & case study & \tabularnewline
\hline 
9. & Study location & & USA, New Mexico, Alamogordo & USA, New Mexico, Alamogordo & \tabularnewline
\hline 
 & \textbf{Cumulative voting} &  &  &  & \tabularnewline
\hline 
10. & What is prioritized? & decision making criteria regarding requirement value  & election candidates & election candidates & software \tabularnewline
\hline 
11. & Number of stakeholders who do CV & 13 & 99 &  & 18\tabularnewline
\hline 
12. & Number of prioritization items,

If CV is used, how many items are in each level? & 13 & 3 &  & 27; 6 high level items, low level groups of 1, 3, 4, 4, 2, and 7 items \tabularnewline
\hline 
13. & How is CV tailored? &  &  & each voter can spend 3 votes on 1, 2 or 3 candidates. & HCV\tabularnewline
\hline 
14. & What methods are used to analyze CV results? & 
chart of final priorities,

PCA - to detect groups of stakeholders with similar priorities &  &  & \tabularnewline
\hline 
 & \textbf{Quality evaluation} &  &  &  & \tabularnewline
\hline 
15. & Study setting rating & 2 & 2 & 3 & 1\tabularnewline
\hline 
16. & Research data availability rating & 0 & 0 & 0 & 1\tabularnewline
\hline 
17. & Rating of correctness of research process & 37 & 36 & 30 & 46\tabularnewline
\hline
\end{tabular}%
}
\end{table}

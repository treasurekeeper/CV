
\begin{table}[ht]
\caption{Extracted data (part 6)}
\scalebox{0.8}{

\begin{tabular}{|>{\raggedright}p{0.03\textwidth}|>{\raggedright}p{0.3\textwidth}|>{\raggedright}p{0.17\textwidth}|>{\raggedright}p{0.17\textwidth}|>{\raggedright}p{0.17\textwidth}|>{\raggedright}p{0.17\textwidth}|}
\hline 
No. & Data item & \multicolumn{4}{l|}{Extracted data}\tabularnewline
\hline
1. & Data extractor & A & A & A & A\tabularnewline
\hline 
2. & Reference & \citet{Berander2006a} & \citet{Berander2004} & \citet{Barney2009b} & \citet{Kuzniarz2010}\tabularnewline
\hline 
3. & Title & Hierarchical Cumulative Voting (HCV) prioritization of requirements
in hierarchies  & Using students as subjects in requirements prioritization & Software Product Quality: Ensuring a Common Goal & Empirical extension of a classification framework for addressing consistency
in model based development \tabularnewline
\hline 
 & \textbf{General information} &  &  &  & \tabularnewline
\hline 
4. & Research area & prioritization in software engineering & software requirements engineering & software quality & model driven development\tabularnewline
\hline 
5. & Study subjects & professionals & students & professionals & professionals\tabularnewline
\hline 
6. & Study setting & industry & academia & industry & industry\tabularnewline
\hline 
7. & Is CV used as research method (research m.) or industry practice (industry
p.) & industry p. & industry p. & industry p. & industry p.\tabularnewline
\hline 
8. & Type of the study & case study & experiment & case study & case study\tabularnewline
\hline 
9. & Study location & & Sweden & & \tabularnewline
\hline 
 & \textbf{Cumulative voting} &  &  &  & \tabularnewline
\hline 
10. & What is prioritized? & software requirement change requests & software requirements & software product qualities & model consistency perspectives and issues\tabularnewline
\hline 
11. & Number of stakeholders who do CV & 19 & 20 & 31 & 24\tabularnewline
\hline 
12. & Number of prioritization items,

If CV is used, how many items are in each level? & 14 & 
18 items. Five separate groups prioritized from 10 up to 18 items.
&
27; 3 high level items, low level groups of 4, 18, and 2 items &
36; 5 high level items, low level groups of 10, 8, 7 and 6 items \tabularnewline
\hline 
13. & How is CV tailored? &  &  & HCV & HCV\tabularnewline
\hline 
14. & What methods are used to analyze CV results? & average divergence of priorities assigned by a stakeholder, average percentage of items given non-zero value &  
&
correlation matrix between stakeholder groups,

correlation coefficient between priorities today and ideal situation,

Spearmans r correlation coefficient is used
 & 
minimum, maximum, mean, and median priority, 

standard deviation of priority,

bar chart of prioritization results showing mean priority and standard deviation of priorities

\tabularnewline
\hline 
 & \textbf{Quality evaluation} &  &  &  & \tabularnewline
\hline 
15. & Study setting rating & 3 & 0 & 3 & 3\tabularnewline
\hline 
16. & Research data availability rating & 2 & 0 & 2 & 0\tabularnewline
\hline 
17. & Rating of correctness of research process & 28 & 33 & 34 & 37\tabularnewline
\hline
\end{tabular}%
}
\end{table}

\section{Background\label{background}}

This section presents definitions and places this study in a context. In the coming sections we will cover: a description of software requirements prioritization methods; examples of CV result analysis methods; and a description of compositional data analysis and CV.

\subsection{Prioritization Methods}

Some of the most popular prioritization methods are the analytical hierarchy
process (AHP), cumulative voting (CV), ranking, numerical assignment,
top-ten, the planning game, minimal spanning tree, bubble sort and binary
search tree \cite{Berander2005,Karlsson1998}. Ranking and numerical
assignment methods perform prioritization on an ordinal scale. AHP and CV
are, on the one hand, considered to be harder to use and also more time 
consuming compared to other methods but, on the other hand, produce 
priorities in ratio scale.

%Prioritization can be used not just to decide which factors to address,
%but also to determine the relative order of importance. In market-driven 
%software development a small part of a very large number of requirements
%need to be selected and divided into several releases to maximize return
%on investment. While in bespoke requirements, focusing on early delivery
%of value can help reduce the risk of project cancellation.
%
Ratio scale priorities have several advantages over ordinal scale
priorities. Ratio scale shows not just the order of items but also
relative distance between them. This enables the priority of a group of
items to be calculated by summing up the priorities of individual items
\cite{Berander2006a}. It is possible to say that one item or set
of items has higher priority than another set of items. Supposing stakeholders
have to choose between several low priority items and one item with
higher priority; with ordinal scale, the item with highest priority will
always be selected first. However, if priorities are given on a ratio
scale, it is possible that lower priority items will be selected if
their cumulative priority is higher. 

Finally, the ratio scale allows the combining of multiple priority factors by calculating ratios
between them. One example of this is the cost-value ratio that shows which
requirements give more value for less money \cite{Karlsson1997}.

\subsection{Prioritization Result Analysis}

%Different studies use and analyze CV in different ways. 
Disagreement between stakeholders happens when two or more stakeholders have assigned
a different priority to one prioritization item. If the level of disagreement
is high it may indicate potential conflicts between stakeholders.
Such conflicts may be of technical character, as well as social or cultural.

The satisfaction a stakeholder has with the final prioritization results is
determined by the difference between the results and the individual priorities
of the stakeholder. A smaller level of difference leads to higher satisfaction.
In the end, stakeholder satisfaction is important because it is necessary to achieve
stakeholder commitment.

In some cases a part of stakeholders may form a group of some kind and, therefore, prioritize
requirements similarly. It may be useful
to detect whether a group of stakeholders has different preferences
compared to other stakeholders. As an example, in \cite{Pettersson2008}, domain experts,
technical experts, managers, project managers, testers, and developers
use CV to prioritize software process improvement issues and the CV results
are analyzed using disagreement charts and satisfaction charts.
Finally, principal component analysis (PCA) is used to identify distinct
groups of stakeholders.

The same items can be prioritized by the same stakeholders multiple times
from different perspectives. In this case it is useful to determine correlation between
the priorities in different perspectives to assess the differences
between the perspectives. As an example, in \cite{Barney2009b}, CV is used by developers,
testers and managers to prioritize quality attributes. The same quality
attributes are prioritized from two perspectives: the perceived situation
today and the perceived ideal situation. Correlation between the two perspectives
is evaluated using the Spearman rank correlation matrix. This allows an analysis of
how well the company balances the priorities of software quality attributes.

In \cite{Jonsson2005} change impact issues are prioritized by developers,
testers, managers, and system architects. The prioritization is done
with respect to three perspectives: strategic, tactical, and operative.
In order to determine correlation between the perspectives, CV results are
analyzed using the Kruskal-Wallis test. In \cite{Chatzipetrou2010} the
results of \cite{Jonsson2005} are further analyzed using PCA, bi-plot, and
ternary plot. In this case, PCA is used to find correlated issues, 
bi-plot shows variance, correlation, difference between
the priorities of issues, and the viewpoints of stakeholders, while ternary
plots are used to show the relative number of issues that received high,
medium, and low priority.

As can be seen above, from the examples above, prioritization has been performed 
with various stakeholders, using different perspectives and, in the end, also analyzed using 
various techniques. We will next describe in more detail one of the more common methods to 
manage prioritization issues---cumulative voting---which has been used in software engineering 
for some time. (CV has its roots in corporate governance and biology.)
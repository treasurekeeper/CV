
\begin{table}
	\center
	\scriptsize
\caption{Extracted data (part 7)}
\scalebox{0.75}{

\begin{tabular}{|>{\raggedright}p{0.03\textwidth}|>{\raggedright}p{0.3\textwidth}|>{\raggedright}p{0.14\textwidth}|>{\raggedright}p{0.17\textwidth}|>{\raggedright}p{0.2\textwidth}|>{\raggedright}p{0.17\textwidth}|}
\hline 
No. & Data item & \multicolumn{4}{l|}{Extracted data}\tabularnewline
\hline
1. & Data extractor & A & A & A & A\tabularnewline
\hline 
2. & Reference & \cite{Bowler2001} &\cite{Rovegard2008} & \cite{Chatzipetrou2010} & \cite{Brockington2003}\tabularnewline
\hline 
3. & Title & Election systems and voter turnout: Experiments in the United States & An Empirical Study on Views of Importance of Change Impact Analysis
Issues & Prioritization of Issues and Requirements by Cumulative Voting: A
Compositional Data Analysis Framework & A Low Information Theory of Ballot Position Effect \tabularnewline
\hline 
 & \textbf{General information} &  &  &  & \tabularnewline
\hline 
4. & Research area & government elections & requirements engineering, change impact analysis & requirements engineering, change impact analysis & government elections\tabularnewline
\hline 
5. & Study subjects & electorate & professionals & professionals & electorate\tabularnewline
\hline 
6. & Study setting & real world & industry & industry & real world\tabularnewline
\hline 
7. & Is CV used as research method (research m.) or industry practice (industry
p.) & industry p. & industry p. & industry p. & industry p.\tabularnewline
\hline 
8. & Type of the study &  & case study & case study & case study\tabularnewline
\hline 
9. & Study location &  & Sweden & Sweden & USA, Illinois, Peoria\tabularnewline
\hline 
 & \textbf{Cumulative voting} &  &  &  & \tabularnewline
\hline 
10. & What is prioritized? & election candidates & Requirement change impact analysis issues & Requirement change impact analysis issues & election candidates\tabularnewline
\hline 
11. & Number of stakeholders who do CV &  &  &  & \tabularnewline
\hline 
12. & Number of prioritization items,

If CV is used, how many items are in each level? &  & 25 & 25 & \tabularnewline	
\hline 
13. & How is CV tailored? &  &  &  & each voter can spend 3 votes on 1, 2 or 3 candidates.\tabularnewline
\hline 
14. & What methods are used to analyze CV results? &  & 
chart for comparing priorities from two perspectives,

bar chart of prioritization results,

difference between priorities assigned  by each two stakeholders using Chi-square statistic,

difference between perspectives using Chi-square statistic

&
PCA - to detect groups of stakeholders with similar priorities,

biplot,

ternary plot 
& \tabularnewline
\hline 
 & \textbf{Quality evaluation} &  &  &  & \tabularnewline
\hline 
15. & Study setting rating & 3 & 3 & 3 & 3\tabularnewline
\hline 
16. & Research data availability rating & 0 & 0 & 0 & 0\tabularnewline
\hline 
17. & Rating of correctness of research process & 29 & 38 & 31  & 28\tabularnewline
\hline
\end{tabular}%
}
\end{table}

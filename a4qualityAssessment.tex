\section{\label{app:QE}Quality Evaluation Checklist}

\begin{table}[h!]
\caption{Quality evaluation checklist}
\scalebox{0.5}{

\begin{tabular}{|>{\raggedright}p{0.04\textwidth}|>{\raggedright}p{0.3\textwidth}|>{\raggedright}p{1.3\textwidth}|>{\raggedright}p{0.09\textwidth}|}
\hline 
 & Item & Question or Description of the Item & Rating\tabularnewline
\hline 
1. & Background, introduction & Introduce research area  & \tabularnewline
\hline 
2. & Problem statement, purpose & What is the problem\citep{Jedlitschka2005}?
Where does it occur \citep{Jedlitschka2005}?

Who has observed it \citep{Jedlitschka2005}?
Why is it important to be solved \citep{Jedlitschka2005}? & \tabularnewline
\hline 
3. & Context, independent variables (aka. environment, setting) & Study location, time constraints, application domain, organization,
tools, market, process (e.g. software development methodology), size
of project, product that is being developed & \tabularnewline
\hline 
4. & Related work & Other existing work, alternative technologies, solutions, and studies & \tabularnewline
\hline 
5. & Goals and Hypotheses & Null hypothesis and one or more alternative hypotheses for each goal & \tabularnewline
\hline 
6. & Research questions &  & \tabularnewline
\hline 
\multicolumn{1}{|l|}{7.} & Design, Research methods &  & \tabularnewline
\cline{1-3}
7.1. & Design & Description of each step of the study & \tabularnewline
\cline{1-3}
7.2. & Control group & If there is a control group, are participants similar to the treatment
group participants in terms of variables that may affect study outcomes\citep{Kitchenham2007}? & \tabularnewline
\cline{1-3}
7.3. & Randomization & Random selection of participants and objects

Random assignment of treatment and objects to participants

Random order of treatments in case of paired design. If each participant
is assigned two treatments A and B, then part of participants perform
A first and the other part start with B & \tabularnewline
\cline{1-3}
7.4. & Blocking & Group participants of the study into homogeneous groups called blocks
(e.g. students in one course, database developers in one company)
and implement the study design within each block independently. The
idea is that variability of independent variables (e.g. experience
and knowledge of subjects) is smaller within a group. That helps measuring
changes in dependent variables \citep{Kitchenham2004}. & \tabularnewline
\cline{1-3}
7.5. & Balancing & Equal number of subjects should be assigned to each treatment \citep{Kitchenham2004}. & \tabularnewline
\cline{1-3}
7.6. & Blinding & Automated assignment of treatments to subjects \citep{Kitchenham2004}

Automated distribution of study materials to subjects \citep{Kitchenham2004}

Persons who grade the task results should not know which treatment
was used \citep{Kitchenham2004}

Analyst should not know which treatment group is which \citep{Kitchenham2004}

Automated data collection from subjects \citep{Kitchenham2004} & \tabularnewline
\hline 
\multicolumn{1}{|l|}{8.} & Subjects (participants) &  & \tabularnewline
\cline{1-3}
8.1. & Population &  & \tabularnewline
\cline{1-3}
8.2. & Sampling & How sampling is performed?

What subjects are included and excluded? \citep{Kitchenham2007}

What is the type of the sampling (e.g. convenience, random)?

Is the sample(selected participants) representative of the population? & \tabularnewline
\cline{1-3}
8.3. & {}``Drop outs'' and response rate & Are reasons given for refusal to participate\citep{Kitchenham2007}? & \tabularnewline
\cline{1-3}
8.4. & Subject motivation & E.g. material benefits, course credits for students, etc. & \tabularnewline
\hline 
9. & Objects & E.g. documents and other artefacts & \tabularnewline
\hline 
10. & Measures, Data collection procedures & Who, when, and how does the measurements \citep{Kitchenham2007}?

How is the measurement supported? Is it automated \citep{Kitchenham2007}?

Are the measures used in the study the most relevant ones for answering
the research questions \citep{Kitchenham2007}? & \tabularnewline
\hline 
\multicolumn{1}{|l|}{11.} & Analysis procedure &  & \tabularnewline
\cline{1-3}
11.1. & Data description & Do the numbers add up across different tables and subgroups \citep{Kitchenham2007}? & \tabularnewline
\cline{1-3}
11.2. & Data types (continuous, ordinal, categorical) &  & \tabularnewline
\cline{1-3} 
11.3. & Scoring systems &  & \tabularnewline
\cline{1-3}
11.4. & Data set reduction, outliers &  & \tabularnewline
\cline{1-3}
11.5. & Statistical methods & Are the assumptions of statistical methods met?

What statistical programs are used? & \tabularnewline
\cline{1-3}
11.6. & Statistical significance & If statistical tests are used to determine differences, is the actual
p value given \citep{Kitchenham2007}?

If the study is concerned with differences among groups, are confidence
limits given describing the magnitude of any observed differences
\citep{Kitchenham2007}? & \tabularnewline
\hline 
12. & Validity threats & Threats, implications of the threats, and threat mitigation & \tabularnewline
\cline{1-3}
12.1. & Side-effects during study execution & Deviations from the plan, solutions for the deviations & \tabularnewline
\hline 
13. & Most important findings  & Are all study questions answered \citep{Kitchenham2007}?

Are negative findings presented \citep{Kitchenham2007}? & \tabularnewline
\hline 
14. & Industry impact, inference, generalisation & What implications does the report have for practice \citep{Kitchenham2007}?

How and where the results can be used?

Limitations under which findings are relevant \citep{Jedlitschka2005}? & \tabularnewline
\hline 
15. & Future work &  & \tabularnewline
\hline

\end{tabular}
} % end scale box
\end{table}